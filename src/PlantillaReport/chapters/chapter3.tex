\documentclass[../main]{subfiles}
\usepackage{lastpage,xr,refcount,etoolbox}
%\externaldocument{../Appendices/Appendix1-CodigosBase}
\begin{document}


\chapter{¿Es ``Unary-Ssum Problem” NPC?}

{
\hypersetup{linkcolor=black}
\minitoc
\vspace{5mm}
}

En este apartado vamos a comprobar cómo la demostración que hemos usado para mostrar que \textbf{Subset-Sum }es \textbf{NPC} falla al intentarlo con \textbf{Unary-Ssum}. \\\\
Realmente no hay que cambiar nada en la demostración, puesto que el problema es exactamemte el mismo (solo cambia la forma en la que se expresan los números). De esta manera, lo más rápido para justificar el por qué esta demostración falla, sería señalar que la demostración construye un conjunto \textbf{S} con los números en decimal, pero para \textbf{Unary-Ssum}, los números deberían ser unarios. \\\\
Eso en teoría sería justificación suficiente, pero por no quedarnos tan limitados, vamos a suponer que podemos cambiar un poco la demostración y nos tomaremos la licencia de una vez construída la tabla, pasar todos los números de \textbf{S}, de decimal a unario.

\section{Complejidad de la transformación}
En el caso del problema \textbf{Subset-Sum}, nuestra reducción tiene una complejidad tal que:\vspace{2mm}
\begin{equation*}
    \textit{nº de variables = \textbf{l}}
\end{equation*}
\begin{equation*}
    \textit{nº de claúsulas = \textbf{k}}
\end{equation*}
\begin{equation*}
    \textit{La tabla tiene (2l+2k) filas * (l+k) columnas}
\end{equation*}
\begin{equation*}
    \textit{l,k} < n 
\end{equation*}
\begin{equation*}
    \textit{La tabla tiene alrededor de (2n+2n) filas * (n+n) columnas}
\end{equation*}
\begin{equation*}
    \textit{La tabla tiene } O(n^2) \textit{ celdas}\vspace{2mm}
\end{equation*} 
Esto demuestra que en tiempo polinómico nuestra función \textbf{f} es capaz de realizar la reducción deseada. Pero en el caso del problema en el que la representación tiene que ser en unario, por cada una de las filas, ya no hay \textbf{n} columnas, sino \textbf{10} elevado a \textbf{n} columnas. Esto quiere decir que la complejidad de la transformación, sigue una función \textbf{t(n)}, tal que:
\begin{equation*}
    t(n) \in O(n*10^n) \rightarrow t(n) \notin O(n^k)\ \  \forall k > 0
\end{equation*}
Como la transformación de la función \textbf{f} para el problema con la representación en unario no se realiza en \textbf{poli-t}, la demoestración de que \textbf{Unary-Ssum} es \textbf{NPC}, falla.

    
\end{document}